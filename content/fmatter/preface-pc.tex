\section*{Message from the Program Committee Co-Chairs}
\setheaders%
    {Message from the Program Committee Co-Chairs}%
    {Message from the Program Committee Co-Chairs}
\thispagestyle{emptyheader}
%\renewcommand{\large}{\fontsize{9}{11}\selectfont}
% that's a hack to make this part nicely fill the pages
\begin{large}

\setlength{\parskip}{.7ex}
%\setlength{\parindent}{0pt}

Welcome to NAACL HLT 2013 in Atlanta, Georgia. We have an exciting program consisting of six tutorials, 24 sessions of talks (both for long and short papers), an insane poster madness session that includes posters from the newly revamped student research workshop, ten workshops and two additional cross-pollination workshops held jointly with ICML (occurring immediately after NAACL HLT, just one block away). There are a few innovations in the conference this year, the most noticeable of which is the twitter channel \#naacl2013 and the fact that we are the first conference to host papers published in the Transactions of the ACL journal -- there are six such papers in our program, marked as [TACL].
We are very excited about our two invited talks, one on Monday morning and one Wednesday morning. The first is by Gina Kuperberg, who will talk about ``Predicting Meaning: What the Brain tells us about the Architecture of Language Comprehension.'' The second presenter is our own Kathleen KcKeown, who will talk about ``Natural Language Applications from Fact to Fiction.''

The morning session on Tuesday includes the presentation of best paper awards to two worthy recipients. The award for Best Short Paper goes to Marta Recasens, Marie-Catherine de Marneffe and Christopher Potts for their paper ``The Life and Death of Discourse Entities: Identifying Singleton Mentions'' The award for Best Student Paper goes to the long paper ``Automatic Generation of English Respellings'' by Bradley Hauer and Greg Kondrak. We gratefully acknowledge IBM’s support for the Student Best Paper Award. Finally, many thanks to the Best Paper Committee for selecting these excellent papers!

The complete program includes 95 long papers (of which six represent presentations from the journal Transactions of the ACL, a first for any ACL conference!) and 51 short papers. We are excited that the conference is able to present such a dynamic array of papers, and would like to thank the authors for their great work. We worked hard to keep the conference to three parallel sessions at any one time to hopefully maximize a participant's ability to see everything she wants! This represents an acceptance rate of 30\% for long papers and 37\% for short papers. More details about the distribution across areas and other statistics will be made available in the NAACL HLT Program Chair report on the ACL wiki: \url{aclweb.org/adminwiki/index.php?title=Reports}

The review process for the conference was double-blind, and included an author response period for
clarifying reviewers’ questions. We were very pleased to have the assistance of 350 reviewers, each of whom reviewed an average of 3.7 papers, in deciding the program. We are especially thankful for the reviewers who spent time reading the author responses and engaging other reviewers in the discussion board. Assigning reviewers would not have been possible without the hard work of Mark Dredze and his miracle assignment scripts. Furthermore, constructing the program would not have been possible without 22 excellent area chairs forming the Senior Program Committee: Eugene Agichtein, Srinivas Bangalore, David Bean, Phil Blunsom, Jordan Boyd-Graber, Marine Carpuat, Joyce Chai, Vera Demberg, Bill Dolan, Doug Downey, Mark Dredze, Markus Dreyer, Sanda Harabagiu, James Henderson, Guy Lapalme, Alon Lavie, Percy Liang, Johanna Moore, Ani Nenkova, Joakim Nivre, Bo Pang, Zak Shafran, David Traum, Peter Turney, and Theresa Wilson. Area chairs were responsible for managing paper assignments, collating reviewer responses, handling papers for other area chairs or program chairs who had conflicts of interest, making recommendations for paper acceptance or rejection, and nominating best papers from their areas. We are very grateful for the time and energy that they have put into the program.

There are a number of other people that we interacted with who deserve a hearty thanks for the success of the program. Rich Gerber and the START team at Softconf have been invaluable for helping us with the mechanics of the reviewing process. Matt Post and Colin Cherry, as publications co-chairs, have been very helpful in assembling the final program and coordinating the publications of the workshop proceedings. There are several crucial parts of the overall program that were the responsibility of various contributors, including Annie Louis, Richard Socher, Julia Hockenmaier and Eric Ringger (Student Research Workshop chairs, who did an amazing job revamping the SRW); Jimmy Lin and Katrin Erk (Tutorial Chairs); Luke Zettlemoyer and Sujith Ravi (Workshop Chairs); Chris Dyer and Derrick Higgins (Demo Chairs); Jacob Eisenstein (Student Volunteer Coordinator); Chris Brew (Local Sponsorship Chair); Patrick Pantel and Dan Bikel (Sponsorship Chairs); and the new-founded Publicity chair who handled \#naacl2013 tweeting among other things, Kristy Boyer.

We would also like to thank Chris Callison-Burch and the NAACL Executive Board for guidance during the process. Michael Collins was amazingly helpful in getting the inaugural TACL papers into the NAACL HLT conference. Priscilla Rasmussen deserves, as always, special mention and warmest thanks as the local arrangements chair and general business manager. Priscilla is amazing and everyone who sees her at the conference should thank her.

Finally, we would like to thank our General Chair, Lucy Vanderwende, for both her trust and guidance during this process. She helped turn the less-than-wonderful parts of this job to roses, and her ability to organize an incredibly complex event is awe inspiring. None of this would have happened without her.

We hope that you enjoy the conference!

\vspace{.2in}
Hal Daumé III, University of Maryland \\
\indent Katrin Kirchhoff, University of Washington

%\cleardoublepage
\end{large}
%\renewcommand{\large}{\fontsize{10}{12}\selectfont}
