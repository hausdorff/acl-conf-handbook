\addcontentsline{toc}{chapter}{Program}
\setlength{\parindent}{0in}
\setlength{\parskip}{2ex}
\renewcommand{\baselinestretch}{0.87}

\begin{center}
{\Large \bf The schedule}
\end{center}
\vspace{3mm}

\begin{tabular}{p{20mm}p{128mm}}
9:00 & Welcome\\

9:00-10:00 & Invited talk 1\\
 & {\em Reflections on Verbal Art 40 years after}\\
 & Livia Polanyi\\

10:00-10:30 & \hyperlink{page.1}{\em A Tale of Two Cultures: Bringing Literary Analysis and Computational Linguistics Together}\\
 & Adam Hammond, Julian Brooke and Graeme Hirst \\

\\
10:30-11:00 & Coffee break\\
\\

11:00-11:30 & \hyperlink{page.9}{\em Recognition of Classical Arabic Poems}\\
 & Abdulrahman Almuhareb, Ibrahim Alkharashi, Lama Al Saud and Haya Altuwaijri \\

11:30-12:00 & \hyperlink{page.17}{\em Tradition and Modernity in 20th Century Chinese Poetry}\\
 & Rob Voigt and Dan Jurafsky \\

12:00-12:30 & \hyperlink{page.23}{\em Linguistic Resources and Topic Models for the Analysis of Persian Poems}\\
 & Ehsaneddin Asgari and Jean-Cedric Chappelier \\

\\
12:30-14:00 & Lunch break\\
\\

14:00-15:00 & Invited talk 2\\
 & {\em Intelligent Narrative Generation: From Cognition to Crowdsourcing}\\
 & Mark Riedl\\

15:00-15:30 & Poster teasers\\
 & \hyperlink{page.42}{\em The desirability of a corpus of online book responses}\\
 & Peter Boot \\

 & \hyperlink{page.51}{\em Clustering Voices in The Waste Land}\\
 & Julian Brooke, Graeme Hirst and Adam Hammond \\

 & \hyperlink{page.66}{\em An initial study of topical poetry segmentation}\\
 & Chris Fournier \\

 & \hyperlink{page.57}{\em Groundhog DAG: Representing Semantic Repetition in Literary Narratives}\\
 & Greg Lessard and Michael Levison \\

 & \hyperlink{page.71}{\em Exploring Cities in Crime: Significant Concordance and Co-occurrence in Quantitative Literary Analysis}\\
 & Janneke Rauscher, Leonard Swiezinski, Martin Riedl and Chris Biemann \\

\\
15:30-16:00 & Coffee break\\
\\

16:00-16:30 & Poster session\\
\\

16:30-17:00 & \hyperlink{page.32}{\em From high heels to weed attics: a syntactic investigation of chick lit and literature}\\
 & Kim Jautze, Corina Koolen, Andreas van Cranenburgh and Hayco de Jong \\

17:00-17:30 & An informal presentation\\
& {\em Identification of Speakers in Novels}\\
 & Hua He, Denilson Barbosa and Greg Kondrak \\

17:30 & Farewell

\end{tabular}

\newpage

\begin{center}
{\large \bf Invited speaker 1}
\end{center}

{\bf Livia Polanyi} joined Stanford University in 2012 as Consulting Professor of Linguistics after leaving Microsoft Corporation where she was a Principal Researcher at Bing working on applications of formal theories of discourse structure to problems in search. She also taught at the University of Amsterdam, Rice University and the University of Tel Aviv, and held scientist positions in computational linguistics at BBN Labs, Fuji-Xerox Palo Alto Labs and Powerset Corporation where she was the first member of the technical team. Professor Polanyi's research focusses on the structure of language above the sentence and she has published work in theoretical, socio and computational linguistics as well as in literary theory, anthropology, economics and political science. Currently she is working on extensions of formal concepts developed to account for discourse interpretability despite discontinuity to foundational problems in music, dance and conversation. She is also a poet.    

\begin{center}
{\Large \em Reflections on Verbal Art 40 years after}
\end{center}

{\large Abstract}

Many years ago, a young girl who dared not call herself a “poet” sat alone at her desk day after day writing texts that she dared not call “poems”. Each text, she knew, was a little theory about the nature of language. Once the text began, she simply did what was required. The artist is servant not mistress, she had learned. Creation is strictly carrying out what one is told.  It was a lonely life. No one saw the words that she wrote.  As a foot soldier in the army of those who wrote for the desk drawer, she talked to herself and grew impatient with what she had to say and so she decided one day to leave her desk and go out into the world beyond the window, to read instead of write and to listen to what others had to say. She wanted to understand what she had been doing day after day at her desk near the window because she knew that what had been happening there was that literature was being born -- not great literature, probably not even good literature – and she wanted to know what this “literature” she was so busy serving might be. And so, she began to study the nature of language and to put off all consideration of what on earth that girl behind the window had been doing. She read and she studied and she learned and eventually she wrote and she explained and she taught and she left behind the questions about the nature of literature that had sent her out in the world to understand.

But the years have a way of catching up with everyone and as summers became winters and winters became summers again, the young girl became a young woman and then a woman no longer young and then again that young woman no longer young found herself alone at a desk writing texts the she dared not call “poems” but now as she wrote as the theories flowed out onto the page she had learned enough to understand what these texts were theories of and why and how the language of the first breath that would grow into a full text determined the possibilities of what that text could become.  And so, in this talk, having been asked to talk to this workshop on computational approaches to literature I will share with you some speculations about the nature of Verbal Art that I have learned through study and practice in the years that separated the woman who will stand before you from that young girl who sat behind her desk near a window years and years ago and wrote down texts that she did not give the name that they had earned. 

\newpage

\begin{center}
{\large \bf Invited speaker 2}
\end{center}

{\bf Mark Riedl} is an Assistant Professor in the Georgia Tech School of Interactive Computing and director of the Entertainment Intelligence Lab. Dr. Riedl's research focuses on the intersection of artificial intelligence, virtual worlds, and storytelling. The principle research question Dr. Riedl addresses through his research is: how can intelligent computational systems reason about and autonomously create engaging experiences for users of virtual worlds and computer games. Dr. Riedl earned a PhD degree in 2004 from North Carolina State University, where he developed intelligent systems for generating stories and managing interactive user experiences in computer games. From 2004 to 2007, Dr. Riedl was a Research Scientist at the University of Southern California Institute for Creative Technologies where he researched and developed interactive, narrative-based training systems. Dr. Riedl joined the Georgia Tech College of Computing in 2007 and in 2011 he received a DARPA Young Faculty Award for his work on artificial intelligence, narrative, and virtual worlds. His research is supported by the NSF, DARPA, the U.S. Army, and Disney.

\begin{center}
{\Large \em Intelligent Narrative Generation: From Cognition to Crowdsourcing}
\end{center}

{\large Abstract}

Storytelling is a pervasive part of the human experience--we as humans tell stories to communicate, inform, entertain, and educate. But what about computational systems? There are many applications for which we would also like intelligent system to reason about, understand, and create narrative structures: from recognition to question-answering, from entertainment to education. In this talk I will look at one particular aspect of computational modeling of narrative: automated story generation, the problem of creating novel narrative fabula event sequences for dramatic, pedagogical, or other purposes. I will trace the evolution of fabula story generation from its roots in cognitive systems to data-driven techniques and crowdsourcing. I will speculate on how systems may eventually learn how to create and tell stories from interacting with humans and literature.