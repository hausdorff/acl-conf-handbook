\addcontentsline{toc}{chapter}{Program}
\setlength{\parindent}{0in}
\setlength{\parskip}{2ex}
\renewcommand{\baselinestretch}{0.87}

\begin{center}
{\Large \bf
  Conference Program Summary
}
\end{center}
\vspace{-10mm}
%\usepackage{graphicx}
\graphicspath{{./}{../templates/}{../templates/logo/}}
\includegraphics[height=1.1\textheight]{starsem-sts-semeval-schedule_summary_130510e1.pdf}
\newpage
\graphicspath{{./}{../templates/}{../templates/logo/}}
\includegraphics[height=1.1\textheight]{starsem-sts-semeval-schedule_summary_130510e2.pdf}
\newpage


\begin{center}
{\Large \bf
  Conference Program
}
\end{center}
\vspace{3mm}
\begin{tabular}{p{20mm}p{128mm}}
\\
\multicolumn{2}{l}{\bf Day 1: Thursday June 13, 2013} \\
\\
 & {\bf *SEM Main Conference and Shared Task Sessions (no SemEval on Day 1)} \\
\\
 & {\bf Session PLN1: (6:30--8:30) *SEM Opening Reception and STS Poster Session (All SemEval attendees are invited)} \\
\\

\multicolumn{2}{l}{\bf Day 2: Friday June 14, 2013} \\
\\
 & {\bf (08:00--08:30) Registration} \\
\\
 & {\bf Session SE1: (08:30--09:30) Session 1} \\
\\
08:30--08:40 & Opening remarks \\
\\
08:40--09:00 & \hyperlink{page.1}{\em SemEval-2013 Task 1: TempEval-3: Evaluating Time Expressions, Events, and Temporal Relations}\\
         & Naushad UzZaman, Hector Llorens, Leon Derczynski, James Allen, Marc Verhagen and James Pustejovsky \\
\\

09:00--09:20 & \hyperlink{page.10}{\em ClearTK-TimeML: A minimalist approach to TempEval 2013}\\
         & Steven Bethard \\
\\

09:20--09:30 & \hyperlink{page.15}{\em HeidelTime: Tuning English and Developing Spanish  Resources for TempEval-3}\\
         & Jannik Str\"{o}tgen, Julian Zell and Michael Gertz \\
\\

 & {\bf Session PLN2: (09:30--10:30) Keynote address: David Forsyth} \\
\\
 & {\bf (10:30--11:00) Coffee Break} \\
\\
 & {\bf Session SE2: (11:00--12:30) Session 2} \\
\\
11:00--11:10 & \hyperlink{page.20}{\em ATT1: Temporal Annotation Using Big Windows and Rich Syntactic and Semantic Features}\\
         & Hyuckchul Jung and Amanda Stent \\
\\

11:10--11:30 & \hyperlink{page.25}{\em Semeval-2013 Task 8: Cross-lingual Textual Entailment for Content Synchronization}\\
         & Matteo Negri, Alessandro Marchetti, Yashar Mehdad, Luisa Bentivogli and Danilo Giampiccolo \\
\\

11:30--11:50 & \hyperlink{page.34}{\em SOFTCARDINALITY: Learning to Identify Directional Cross-Lingual Entailment from Cardinalities and SMT}\\
         & Sergio Jimenez, Claudia Becerra and Alexander Gelbukh \\
\\

11:50--12:10 & \hyperlink{page.39}{\em SemEval-2013 Task 5: Evaluating Phrasal Semantics}\\
         & Ioannis Korkontzelos, Torsten Zesch, Fabio Massimo Zanzotto and Chris Biemann \\
\\



\end{tabular}
\newpage
\begin{tabular}{p{20mm}p{138mm}}
\\
\multicolumn{2}{l}{\bf Day 2: Friday June 14, 2013 (continued)} \\\\
12:10--12:30 & \hyperlink{page.48}{\em HsH: Estimating Semantic Similarity of Words and Short Phrases with Frequency Normalized Distance Measures}\\
         & Christian Wartena \\
\\

 & {\bf Session SP1: (12:30--13:30) Lunch Break + Poster Session 1 for Tasks 1, 5, 8} \\
\\

& \hyperlink{page.1}{\em SemEval-2013 Task 1: TempEval-3: Evaluating Time Expressions, Events, and Temporal Relations}\\
         & Naushad UzZaman, Hector Llorens, Leon Derczynski, James Allen, Marc Verhagen and James Pustejovsky \\
\\

& \hyperlink{page.10}{\em ClearTK-TimeML: A minimalist approach to TempEval 2013}\\
         & Steven Bethard \\
\\

 
& \hyperlink{page.53}{\em ManTIME: Temporal expression identification and normalization in the TempEval-3 challenge}\\
         & Michele Filannino, Gavin Brown and Goran Nenadic \\
\\

& \hyperlink{page.15}{\em HeidelTime: Tuning English and Developing Spanish  Resources for TempEval-3}\\
         & Jannik Str\"{o}tgen, Julian Zell and Michael Gertz \\
\\

 & \hyperlink{page.58}{\em FSS-TimEx for TempEval-3: Extracting Temporal Information from Text}\\
         & Vanni Zavarella and Hristo Tanev \\
\\

 & \hyperlink{page.20}{\em ATT1: Temporal Annotation Using Big Windows and Rich Syntactic and Semantic Features}\\
         & Hyuckchul Jung and Amanda Stent \\

\\
& \hyperlink{page.64}{\em JU\_CSE: A CRF Based Approach to Annotation of Temporal Expression, Event and Temporal Relations}\\
         &  Anup Kumar Kolya, Amitava Kundu, Rajdeep Gupta, Asif Ekbal, Sivaji Bandyopadhyay\\
\\

 & \hyperlink{page.73}{\em NavyTime: Event and Time Ordering from Raw Text}\\
         & Nate Chambers \\
\\

 & \hyperlink{page.78}{\em SUTime: Evaluation in TempEval-3}\\
         & Angel Chang and Christopher D. Manning\\
\\

 & \hyperlink{page.83}{\em KUL: Data-driven Approach to Temporal Parsing of Newswire Articles}\\
         & Oleksandr Kolomiyets and Marie-Francine Moens \\
\\
 & \hyperlink{page.88}{\em UTTime: Temporal Relation Classification using Deep Syntactic Features}\\
         & Natsuda Laokulrat, Makoto Miwa, Yoshimasa Tsuruoka and Takashi Chikayama \\
\\
& \hyperlink{page.39}{\em SemEval-2013 Task 5: Evaluating Phrasal Semantics}\\
         & Ioannis Korkontzelos, Torsten Zesch, Fabio Massimo Zanzotto and Chris Biemann \\
\\

\end{tabular}
\newpage
\begin{tabular}{p{20mm}p{138mm}}
\\
\multicolumn{2}{l}{\bf Day 2: Friday June 14, 2013 (continued)} \\\\


& \hyperlink{page.48}{\em HsH: Estimating Semantic Similarity of Words and Short Phrases with Frequency Normalized Distance Measures}\\
         & Christian Wartena \\
\\

 & \hyperlink{page.93}{\em UMCC\_DLSI-(EPS): Paraphrases Detection Based on Semantic Distance}\\
         & H\'{e}ctor D\'{a}vila, Antonio Fern\'{a}ndez Orqu\'{i}n, Alexander Ch\'{a}vez, Yoan Guti\'{e}rrez, Armando Collazo, Jos\'{e} I. Abreu, Andr\'{e}s Montoyo and Rafael Mu\~{n}oz \\
\\

 & \hyperlink{page.98}{\em MELODI: Semantic Similarity of Words and Compositional Phrases using Latent Vector Weighting}\\
         & Tim Van de Cruys, Stergos Afantenos and Philippe Muller \\
\\

 & \hyperlink{page.103}{\em IIRG: A Naive Approach to Evaluating Phrasal Semantics}\\
         & Lorna Byrne, Caroline Fenlon and John Dunnion \\
\\

 & \hyperlink{page.108}{\em ClaC: Semantic Relatedness of Words and Phrases}\\
         & Reda Siblini and Leila Kosseim \\
\\

 & \hyperlink{page.114}{\em UNAL: Discriminating between Literal and Figurative Phrasal Usage Using Distributional Statistics and POS tags}\\
         & Sergio Jimenez, Claudia Becerra and Alexander Gelbukh \\
\\

& \hyperlink{page.25}{\em Semeval-2013 Task 8: Cross-lingual Textual Entailment for Content Synchronization}\\
         & Matteo Negri, Alessandro Marchetti, Yashar Mehdad, Luisa Bentivogli and Danilo Giampiccolo \\
\\

 & \hyperlink{page.118}{\em ECNUCS: Recognizing Cross-lingual Textual Entailment Using Multiple Text Similarity and Text Difference Measures}\\
         & Jiang Zhao, Man Lan and Zheng-Yu Niu \\
\\

 & \hyperlink{page.124}{\em BUAP: N-gram based Feature Evaluation for the Cross-Lingual Textual Entailment Task}\\
         & Darnes Vilari\~{n}o, David Pinto, Saul Le\'{o}n, Yuridiana Aleman and Helena G\'{o}mez \\
\\

 & \hyperlink{page.128}{\em ALTN: Word Alignment Features for Cross-lingual Textual Entailment}\\
         & Marco Turchi and Matteo Negri \\
\\

& \hyperlink{page.34}{\em SOFTCARDINALITY: Learning to Identify Directional Cross-Lingual Entailment from Cardinalities and SMT}\\
         & Sergio Jimenez, Claudia Becerra and Alexander Gelbukh \\
\\


 & \hyperlink{page.133}{\em Umelb: Cross-lingual Textual Entailment with Word Alignment and String Similarity Features}\\
         & Yvette Graham, Bahar Salehi and Timothy Baldwin \\
\\

\end{tabular}
\newpage
\begin{tabular}{p{20mm}p{138mm}}
\\
\multicolumn{2}{l}{\bf Day 2: Friday June 14, 2013 (continued)} \\\\
 & {\bf Session PLN3: (13:30--14:30) Joint Panel: Future of *SEM / STS Shared Task / SemEval} \\
\\
 & {\bf Session SE3: (14:30--15:30) Session 3} \\
\\

14:30--14:50 & \hyperlink{page.114}{\em UNAL: Discriminating between Literal and Figurative Phrasal Usage Using Distributional Statistics and POS tags}\\
         & Sergio Jimenez, Claudia Becerra and Alexander Gelbukh \\
\\

14:50--15:10 & \hyperlink{page.138}{\em SemEval-2013 Task 4: Free Paraphrases of Noun Compounds}\\
         & Iris Hendrickx, Zornitsa Kozareva, Preslav Nakov, Diarmuid \'{O} S\'{e}aghdha, Stan Szpakowicz and Tony Veale \\
\\

15:10--15:30 & \hyperlink{page.144}{\em MELODI: A Supervised Distributional Approach for Free Paraphrasing of Noun Compounds}\\
         & Tim Van de Cruys, Stergos Afantenos and Philippe Muller \\
\\

 & {\bf Session SP2: (15:30--16:30) Coffee Break + Poster Session 2 for Tasks 4, 10, 11, 12} \\
\\

& \hyperlink{page.138}{\em SemEval-2013 Task 4: Free Paraphrases of Noun Compounds}\\
         & Iris Hendrickx, Zornitsa Kozareva, Preslav Nakov, Diarmuid \'{O} S\'{e}aghdha, Stan Szpakowicz and Tony Veale \\
\\

 & \hyperlink{page.148}{\em SFS-TUE: Compound Paraphrasing with a Language Model and Discriminative Reranking}\\
         & Yannick Versley \\
\\

 & \hyperlink{page.153}{\em IIIT-H: A Corpus-Driven Co-occurrence Based Probabilistic Model for Noun Compound Paraphrasing}\\
         & Nitesh Surtani, Arpita Batra, Urmi Ghosh and Soma Paul \\
\\

& \hyperlink{page.144}{\em MELODI: A Supervised Distributional Approach for Free Paraphrasing of Noun Compounds}\\
         & Tim Van de Cruys, Stergos Afantenos and Philippe Muller \\
\\

 & \hyperlink{page.158}{\em SemEval-2013 Task 10: Cross-lingual Word Sense Disambiguation}\\
         & Els Lefever and V\'{e}ronique Hoste \\
\\

 & \hyperlink{page.167}{\em XLING: Matching Query Sentences to a Parallel Corpus using Topic Models for WSD}\\
         & Liling Tan and Francis Bond \\
\\

 & \hyperlink{page.171}{\em HLTDI: CL-WSD Using Markov Random Fields for SemEval-2013 Task 10}\\
         & Alex Rudnick, Can Liu and Michael Gasser \\
\\


\end{tabular}
\newpage
\begin{tabular}{p{20mm}p{138mm}}
\\
\multicolumn{2}{l}{\bf Day 2: Friday June 14, 2013 (continued)} \\\\
 & \hyperlink{page.178}{\em LIMSI : Cross-lingual Word Sense Disambiguation using Translation Sense Clustering}\\
         & Marianna Apidianaki \\
\\

 & \hyperlink{page.183}{\em WSD2: Parameter optimisation for Memory-based Cross-Lingual Word-Sense Disambiguation}\\
         & Maarten van Gompel and Antal van den Bosch \\
\\

 & \hyperlink{page.188}{\em NRC: A Machine Translation Approach to Cross-Lingual Word Sense Disambiguation (SemEval-2013 Task 10)}\\
         & Marine Carpuat \\
\\

 & \hyperlink{page.193}{\em SemEval-2013 Task 11: Word Sense Induction and Disambiguation within an End-User Application}\\
         & Roberto Navigli and Daniele Vannella \\
\\

 & \hyperlink{page.202}{\em Duluth : Word Sense Induction Applied to Web Page Clustering}\\
         & Ted Pedersen \\
\\

 & \hyperlink{page.207}{\em SATTY : Word Sense Induction Application in Web Search Clustering}\\
         & Satyabrata Behera, Upasana Gaikwad, Ramakrishna Bairi and Ganesh Ramakrishnan \\
\\

 & \hyperlink{page.212}{\em UKP-WSI: UKP Lab Semeval-2013 Task 11 System Description}\\
         & Hans-Peter Zorn and Iryna Gurevych \\
\\

 & \hyperlink{page.217}{\em unimelb: Topic Modelling-based Word Sense Induction for Web Snippet Clustering}\\
         & Jey Han Lau, Paul Cook and Timothy Baldwin \\
\\

 & \hyperlink{page.222}{\em SemEval-2013 Task 12: Multilingual Word Sense Disambiguation}\\
         & Roberto Navigli, David Jurgens and Daniele Vannella \\
\\

 & \hyperlink{page.232}{\em GETALP System : Propagation of a Lesk Measure through an Ant Colony Algorithm}\\
         & Didier Schwab, Andon Tchechmedjiev, J\'{e}r\^{o}me Goulian, Mohammad Nasiruddin, Gilles S\'{e}rasset and Herv\'{e} Blanchon \\
\\

 & \hyperlink{page.241}{\em UMCC\_DLSI: Reinforcing a Ranking Algorithm with Sense Frequencies and Multidimensional Semantic Resources to solve Multilingual Word Sense Disambiguation}\\
         & Yoan Guti\'{e}rrez, Yenier Casta\~{n}eda, Andy Gonz\'{a}lez, Rainel Estrada, Dennys D. Piug, Jose I. Abreu, Roger P\'{e}rez, Antonio Fern\'{a}ndez Orqu\'{i}n, Andr\'{e}s Montoyo, Rafael Mu\~{n}oz and Franc Camara \\
\\

 & \hyperlink{page.250}{\em DAEBAK!: Peripheral Diversity for Multilingual Word Sense Disambiguation}\\
         & Steve L. Manion,  and Raazesh Sainudiin \\
\\

\end{tabular}
\newpage
\begin{tabular}{p{20mm}p{138mm}}
\\
\multicolumn{2}{l}{\bf Day 2: Friday June 14, 2013 (continued)} \\\\
 & {\bf Session SE4: (16:30--18:30) Session 4} \\
\\



16:30--16:50  & \hyperlink{page.158}{\em SemEval-2013 Task 10: Cross-lingual Word Sense Disambiguation}\\
         & Els Lefever and V\'{e}ronique Hoste \\
\\

16:50--17:10  & \hyperlink{page.171}{\em HLTDI: CL-WSD Using Markov Random Fields for SemEval-2013 Task 10}\\
         & Alex Rudnick, Can Liu and Michael Gasser \\
\\

17:10--17:30 & \hyperlink{page.193}{\em SemEval-2013 Task 11: Word Sense Induction and Disambiguation within an End-User Application}\\
         & Roberto Navigli and Daniele Vannella \\
\\

17:30--17:50 & \hyperlink{page.217}{\em unimelb: Topic Modelling-based Word Sense Induction for Web Snippet Clustering}\\
         & Jey Han Lau, Paul Cook and Timothy Baldwin \\
\\


17:50--18:10  & \hyperlink{page.222}{\em SemEval-2013 Task 12: Multilingual Word Sense Disambiguation}\\
         & Roberto Navigli, David Jurgens and Daniele Vannella \\
\\


18:10--18:20  & \hyperlink{page.241}{\em UMCC\_DLSI: Reinforcing a Ranking Algorithm with Sense Frequencies and Multidimensional Semantic Resources to solve Multilingual Word Sense Disambiguation}\\
         & Yoan Guti\'{e}rrez, Yenier Casta\~{n}eda, Andy Gonz\'{a}lez, Rainel Estrada, Dennys D. Piug, Jose I. Abreu, Roger P\'{e}rez, Antonio Fern\'{a}ndez Orqu\'{i}n, Andr\'{e}s Montoyo, Rafael Mu\~{n}oz and Franc Camara \\
\\

18:20--18:30  & \hyperlink{page.250}{\em DAEBAK!: Peripheral Diversity for Multilingual Word Sense Disambiguation}\\
         & Steve L. Manion,  and Raazesh Sainudiin \\
\\

 & {\bf \newpage} \\
\\
\multicolumn{2}{l}{\bf Day 3: Saturday June 15, 2013} \\
\\
 & {\bf Session SE5: (08:40--10:30) Session 5} \\
\\
08:40--09:00 & \hyperlink{page.255}{\em SemEval-2013 Task 3: Spatial Role Labeling}\\
         & Oleksandr Kolomiyets, Parisa Kordjamshidi, Marie-Francine Moens and Steven Bethard \\
\\

09:00--09:20 & \hyperlink{page.263}{\em SemEval-2013 Task 7: The Joint Student Response Analysis and 8th Recognizing Textual Entailment Challenge}\\
         & Myroslava Dzikovska, Rodney Nielsen, Chris Brew, Claudia Leacock, Danilo Giampiccolo, Luisa Bentivogli, Peter Clark, Ido Dagan and Hoa Trang Dang \\
\\

09:20--09:35 & \hyperlink{page.275}{\em ETS: Domain Adaptation and Stacking for Short Answer Scoring}\\
         & Michael Heilman and Nitin Madnani \\
\\


\end{tabular}
\newpage
\begin{tabular}{p{20mm}p{138mm}}
\\
\multicolumn{2}{l}{\bf Day 3: Saturday June 15, 2013 (continued)} \\\\
09:35--09:50 & \hyperlink{page.280}{\em SOFTCARDINALITY: Hierarchical Text Overlap for Student Response Analysis}\\
         & Sergio Jimenez, Claudia Becerra and Alexander Gelbukh \\
\\

09:50--10:00 & \hyperlink{page.285}{\em UKP-BIU: Similarity and Entailment Metrics for Student Response Analysis}\\
         & Omer Levy, Torsten Zesch, Ido Dagan and Iryna Gurevych \\
\\

10:00--10:20 & \hyperlink{page.290}{\em SemEval-2013 Task 13: Word Sense Induction for Graded and Non-Graded Senses}\\
         & David Jurgens and Ioannis Klapaftis \\
\\

10:20--10:30 & \hyperlink{page.300}{\em AI-KU: Using Substitute Vectors and Co-Occurrence Modeling For Word Sense Induction and Disambiguation}\\
         & Osman Baskaya, Enis Sert, Volkan Cirik and Deniz Yuret \\
\\

 & {\bf (10:30--11:00) Coffee Break} \\
\\
 & {\bf Session SE6: (11:00--13:10) Session 6} \\
\\
11:00--11:10 & \hyperlink{page.307}{\em unimelb: Topic Modelling-based Word Sense Induction}\\
         & Jey Han Lau, Paul Cook and Timothy Baldwin \\
\\

11:10--11:30 & \hyperlink{page.312}{\em SemEval-2013 Task 2: Sentiment Analysis in Twitter}\\
         & Preslav Nakov, Sara Rosenthal, Zornitsa Kozareva, Veselin Stoyanov, Alan Ritter and Theresa Wilson \\
\\

11:30--11:50 & \hyperlink{page.321}{\em NRC-Canada: Building the State-of-the-Art in Sentiment Analysis of Tweets}\\
         & Saif Mohammad, Svetlana Kiritchenko and Xiaodan Zhu \\
\\

11:50--12:00 & \hyperlink{page.328}{\em GU-MLT-LT: Sentiment Analysis of Short Messages using Linguistic Features and Stochastic Gradient Descent}\\
         & Tobias G\"{u}nther and Lenz Furrer \\
\\

12:00--12:10 & \hyperlink{page.333}{\em AVAYA: Sentiment Analysis on Twitter with Self-Training and Polarity Lexicon Expansion}\\
         & Lee Becker, George Erhart, David Skiba and Valentine Matula \\
\\

12:10--12:30 & \hyperlink{page.341}{\em SemEval-2013 Task 9 : Extraction of Drug-Drug Interactions from Biomedical Texts (DDIExtraction 2013)}\\
         & Isabel Segura-Bedmar, Paloma Mart\'{i}nez and Mar\'{i}a Herrero Zazo \\
\\

12:30--12:50 & \hyperlink{page.351}{\em FBK-irst : A Multi-Phase Kernel Based Approach for Drug-Drug Interaction Detection and Classification that Exploits Linguistic Information}\\
         & Md. Faisal Mahbub Chowdhury and Alberto Lavelli \\
\\

12:50--13:10 & \hyperlink{page.356}{\em WBI-NER: The impact of domain-specific features on the performance of identifying and classifying mentions of drugs}\\
         & Tim Rockt\"{a}schel, Torsten Huber, Michael Weidlich and Ulf Leser \\
\\

\end{tabular}
\newpage
\begin{tabular}{p{20mm}p{138mm}}
\\
\multicolumn{2}{l}{\bf Day 3: Saturday June 15, 2013 (continued)} \\\\
 & {\bf Session SP3: (13:10--15:30) Lunch Break  + Poster Session 3 for Tasks 2, 3, 7, 9, 13} \\
\\
 
& \hyperlink{page.312}{\em SemEval-2013 Task 2: Sentiment Analysis in Twitter}\\
         & Preslav Nakov, Sara Rosenthal, Zornitsa Kozareva, Veselin Stoyanov, Alan Ritter and Theresa Wilson \\
\\


& \hyperlink{page.364}{\em AMI\&ERIC: How to Learn with Naive Bayes and Prior Knowledge: an Application to Sentiment Analysis}\\
         & Mohamed Dermouche, Leila Khouas, Julien Velcin and Sabine Loudcher \\
\\

 & \hyperlink{page.369}{\em UNITOR: Combining Syntactic and Semantic Kernels for Twitter Sentiment Analysis}\\
         & Giuseppe Castellucci, Simone Filice, Danilo Croce and Roberto Basili \\
\\


& \hyperlink{page.328}{\em GU-MLT-LT: Sentiment Analysis of Short Messages using Linguistic Features and Stochastic Gradient Descent}\\
         & Tobias G\"{u}nther and Lenz Furrer \\
\\


& \hyperlink{page.333}{\em AVAYA: Sentiment Analysis on Twitter with Self-Training and Polarity Lexicon Expansion}\\
         & Lee Becker, George Erhart, David Skiba and Valentine Matula \\
\\


 & \hyperlink{page.375}{\em TJP: Using Twitter to Analyze the Polarity of Contexts}\\
         & Tawunrat Chalothorn and Jeremy Ellman \\
\\

 & \hyperlink{page.380}{\em uOttawa: System description for SemEval 2013 Task 2 Sentiment Analysis in Twitter}\\
         & Hamid Poursepanj, Josh Weissbock and Diana Inkpen \\
\\

 & \hyperlink{page.384}{\em UT-DB: An Experimental Study on Sentiment Analysis in Twitter}\\
         & Zhemin Zhu, Djoerd Hiemstra, Peter Apers and Andreas Wombacher \\
\\

 & \hyperlink{page.390}{\em USNA: A Dual-Classifier Approach to Contextual Sentiment Analysis}\\
         & Ganesh Harihara, Eugene Yang and Nate Chambers \\
\\

 & \hyperlink{page.395}{\em KLUE: Simple and robust methods for polarity classification}\\
         & Thomas Proisl, Paul Greiner, Stefan Evert and Besim Kabashi \\
\\

 & \hyperlink{page.402}{\em SINAI: Machine Learning and Emotion of the Crowd for Sentiment Analysis in Microblogs}\\
         & Eugenio Mart\'{i}nez-C\'{a}mara, Arturo Montejo-R\'{a}ez, M. Teresa Mart\'{i}n-Valdivia and L. Alfonso Ure\~{n}a-L\'{o}pez \\
\\

 & \hyperlink{page.408}{\em ECNUCS: A Surface Information Based System Description of Sentiment Analysis in Twitter in the SemEval-2013 (Task 2)}\\
         & Zhu Tiantian, Zhang Fangxi and Man Lan \\
\\

 & \hyperlink{page.414}{\em Umigon: sentiment analysis for tweets based on terms lists and heuristics}\\
         & Clement Levallois \\
\\


\end{tabular}
\newpage
\begin{tabular}{p{20mm}p{138mm}}
\\
\multicolumn{2}{l}{\bf Day 3: Saturday June 15, 2013 (continued)} \\\\
 & \hyperlink{page.418}{\em [LVIC-LIMSI]: Using Syntactic Features and Multi-polarity Words for Sentiment Analysis in Twitter}\\
         & Morgane Marchand, Alexandru Ginsca, Romaric Besan\c{c}on and Olivier Mesnard \\
\\

 & \hyperlink{page.425}{\em SwatCS: Combining simple classifiers with estimated accuracy}\\
         & Sam Clark and Rich Wicentwoski \\
\\

 & \hyperlink{page.430}{\em NTNU: Domain Semi-Independent Short Message Sentiment Classification}\\
         & {\O}yvind Selmer, Mikael Brevik, Bj\"{o}rn Gamb\"{a}ck and Lars Bungum \\
\\

 & \hyperlink{page.438}{\em SAIL: A hybrid approach to sentiment analysis}\\
         & Nikolaos Malandrakis, Abe Kazemzadeh, Alexandros Potamianos and Shrikanth Narayanan \\
\\

 & \hyperlink{page.443}{\em UMCC\_DLSI-(SA): Using a ranking algorithm and informal features to solve Sentiment Analysis in Twitter}\\
         & Yoan Guti\'{e}rrez, Andy Gonz\'{a}lez, Roger P\'{e}rez, Jos\'{e} I. Abreu, Antonio Fern\'{a}ndez Orqu\'{i}n, Alejandro Mosquera, Andr\'{e}s Montoyo, Rafael Mu\~{n}oz and Franc Camara \\
\\

 & \hyperlink{page.450}{\em ASVUniOfLeipzig: Sentiment Analysis in Twitter using Data-driven Machine Learning Techniques}\\
         & Robert Remus \\
\\

 & \hyperlink{page.455}{\em Experiments with DBpedia, WordNet and SentiWordNet as resources for sentiment analysis in micro-blogging}\\
         & Hussam Hamdan, Frederic B\'{e}chet and Patrice Bellot \\
\\

 & \hyperlink{page.460}{\em OPTWIMA: Comparing Knowledge-rich and Knowledge-poor Approaches for Sentiment Analysis in Short Informal Texts}\\
         & Alexandra Balahur \\
\\

 & \hyperlink{page.466}{\em FBK: Sentiment Analysis in Twitter with Tweetsted}\\
         & Md. Faisal Mahbub Chowdhury, Marco Guerini, Sara Tonelli and Alberto Lavelli \\
\\

 & \hyperlink{page.471}{\em SU-Sentilab : A Classification System for Sentiment Analysis in Twitter}\\
         & Gizem Gezici, Rahim Dehkharghani, Berrin Yanikoglu, Dilek Tapucu and Yucel Saygin \\
\\

 & \hyperlink{page.478}{\em Columbia NLP: Sentiment Detection of Subjective Phrases in Social Media}\\
         & Sara Rosenthal and Kathy McKeown \\
\\

 & \hyperlink{page.483}{\em FBM: Combining lexicon-based ML and heuristics for Social Media Polarities}\\
         & Carlos Rodriguez-Penagos, Jordi Atserias Batalla, Joan Codina-Filb\`{a}, David Garc\'{i}a-Narbona, Jens Grivolla, Patrik Lambert and Roser Saur\'{i} \\
\\



\end{tabular}
\newpage
\begin{tabular}{p{20mm}p{138mm}}
\\
\multicolumn{2}{l}{\bf Day 3: Saturday June 15, 2013 (continued)} \\\\
 & \hyperlink{page.490}{\em REACTION: A naive machine learning approach for sentiment classification}\\
         & Silvio Moreira, Jo\~{a}o Filgueiras, Bruno Martins, Francisco Couto and M\'{a}rio J. Silva \\
\\

 & \hyperlink{page.495}{\em IITB-Sentiment-Analysts: Participation in Sentiment Analysis in Twitter SemEval 2013 Task}\\
         & Karan Chawla, Ankit Ramteke and Pushpak Bhattacharyya \\
\\
 & \hyperlink{page.501}{\em SSA-UO: Unsupervised Sentiment Analysis in Twitter}\\
         & Reynier Ortega Bueno, Adrian Fonseca Bruz\'{o}n, Yoan Guti\'{e}rrez and Andres Montoyo \\
\\

 & \hyperlink{page.508}{\em senti.ue-en: an approach for informally written short texts in SemEval-2013 Sentiment Analysis task}\\
         & Jos\'{e} Saias and Hil\'{a}rio Fernandes \\
\\

 & \hyperlink{page.513}{\em teragram: Rule-based detection of sentiment phrases using SAS Sentiment Analysis}\\
         & Hilke Reckman, Cheyanne Baird, Jean Crawford, Richard Crowell, Linnea Micciulla, Saratendu Sethi and Fruzsina Veress \\
\\

 & \hyperlink{page.520}{\em CodeX: Combining an SVM Classifier and Character N-gram Language Models for Sentiment Analysis on Twitter Text}\\
         & Qi Han, Junfei Guo and Hinrich Schuetze \\
\\

 & \hyperlink{page.525}{\em sielers : Feature Analysis and Polarity Classification of Expressions from Twitter and SMS Data}\\
         & Harshit Jain, Aditya Mogadala and Vasudeva Varma \\
\\

 & \hyperlink{page.530}{\em Kea: Expression-level Sentiment Analysis from Twitter Data}\\
         & Ameeta Agrawal and Aijun An \\
\\


& \hyperlink{page.321}{\em NRC-Canada: Building the State-of-the-Art in Sentiment Analysis of Tweets}\\
         & Saif Mohammad, Svetlana Kiritchenko and Xiaodan Zhu \\
\\


 & \hyperlink{page.535}{\em UoM: Using Explicit Semantic Analysis for Classifying Sentiments}\\
         & Sapna Negi and Michael Rosner \\
\\

 & \hyperlink{page.539}{\em bwbaugh : Hierarchical sentiment analysis with partial self-training}\\
         & Wesley Baugh \\
\\

 & \hyperlink{page.543}{\em Serendio: Simple and Practical lexicon based approach to Sentiment Analysis}\\
         & Prabu palanisamy, Vineet Yadav and Harsha Elchuri \\
\\

 & \hyperlink{page.549}{\em SZTE-NLP: Sentiment Detection on Twitter Messages}\\
         & Viktor Hangya, Gabor Berend and Rich\'{a}rd Farkas \\
\\


\end{tabular}
\newpage
\begin{tabular}{p{20mm}p{138mm}}
\\
\multicolumn{2}{l}{\bf Day 3: Saturday June 15, 2013 (continued)} \\\\

 & \hyperlink{page.554}{\em BOUNCE: Sentiment Classification in Twitter using Rich Feature Sets}\\
         & Nadin K\"{o}kciyan, Arda \c{C}elebi, Arzucan \"{O}zg\"{u}r and Suzan \"{U}sk\"{u}darlı \\
\\

 & \hyperlink{page.562}{\em nlp.cs.aueb.gr: Two Stage Sentiment Analysis}\\
         & Prodromos Malakasiotis, Rafael Michael Karampatsis, Konstantina Makrynioti and John Pavlopoulos \\
\\

 & \hyperlink{page.568}{\em NILC\_USP: A Hybrid System for Sentiment Analysis in Twitter Messages}\\
         & Pedro Balage Filho and Thiago Pardo \\
\\

& \hyperlink{page.255}{\em SemEval-2013 Task 3: Spatial Role Labeling}\\
         & Oleksandr Kolomiyets, Parisa Kordjamshidi, Marie-Francine Moens and Steven Bethard \\
\\


 & \hyperlink{page.573}{\em UNITOR-HMM-TK: Structured Kernel-based learning for Spatial Role Labeling}\\
         & Emanuele Bastianelli, Danilo Croce, Roberto Basili and Daniele Nardi \\
\\


& \hyperlink{page.263}{\em SemEval-2013 Task 7: The Joint Student Response Analysis and 8th Recognizing Textual Entailment Challenge}\\
         & Myroslava Dzikovska, Rodney Nielsen, Chris Brew, Claudia Leacock, Danilo Giampiccolo, Luisa Bentivogli, Peter Clark, Ido Dagan and Hoa Trang Dang \\
\\


& \hyperlink{page.285}{\em UKP-BIU: Similarity and Entailment Metrics for Student Response Analysis}\\
         & Omer Levy, Torsten Zesch, Ido Dagan and Iryna Gurevych \\
\\


& \hyperlink{page.275}{\em ETS: Domain Adaptation and Stacking for Short Answer Scoring}\\
         & Michael Heilman and Nitin Madnani \\
\\


 & \hyperlink{page.580}{\em EHU-ALM: Similarity-Feature Based Approach for Student Response Analysis}\\
         & Itziar Aldabe, Montse Maritxalar and Oier Lopez de Lacalle \\
\\

 & \hyperlink{page.585}{\em CNGL: Grading Student Answers by Acts of Translation}\\
         & Ergun Bicici and Josef van Genabith \\
\\

 & \hyperlink{page.592}{\em Celi: EDITS and Generic Text Pair Classification}\\
         & Milen Kouylekov, Luca Dini, Alessio Bosca and Marco Trevisan \\
\\

 & \hyperlink{page.598}{\em LIMSIILES: Basic English Substitution for Student Answer Assessment at SemEval 2013}\\
         & Martin Gleize and Brigitte Grau \\
\\

& \hyperlink{page.280}{\em SOFTCARDINALITY: Hierarchical Text Overlap for Student Response Analysis}\\
         & Sergio Jimenez, Claudia Becerra and Alexander Gelbukh \\
\\


 & \hyperlink{page.603}{\em CU : Computational Assessment of Short Free Text Answers - A Tool for Evaluating Students' Understanding}\\
         & IFEYINWA OKOYE, Steven Bethard and Tamara Sumner \\
\\


\end{tabular}
\newpage
\begin{tabular}{p{20mm}p{138mm}}
\\
\multicolumn{2}{l}{\bf Day 3: Saturday June 15, 2013 (continued)} \\\\
 & \hyperlink{page.608}{\em CoMeT: Integrating different levels of linguistic modeling for meaning assessment}\\
         & Niels Ott, Ramon Ziai, Michael Hahn and Detmar Meurers \\
\\

& \hyperlink{page.341}{\em SemEval-2013 Task 9 : Extraction of Drug-Drug Interactions from Biomedical Texts (DDIExtraction 2013)}\\
         & Isabel Segura-Bedmar, Paloma Mart\'{i}nez and Mar\'{i}a Herrero Zazo \\
\\

 & \hyperlink{page.617}{\em UC3M: A kernel-based approach to identify and classify DDIs in bio-medical texts.}\\
         & Daniel Sanchez-Cisneros \\
\\

 & \hyperlink{page.622}{\em UEM-UC3M: An Ontology-based named entity recognition system for biomedical texts.}\\
         & Daniel Sanchez-Cisneros and Fernando Aparicio Gali \\
\\

& \hyperlink{page.351}{\em FBK-irst : A Multi-Phase Kernel Based Approach for Drug-Drug Interaction Detection and Classification that Exploits Linguistic Information}\\
         & Md. Faisal Mahbub Chowdhury and Alberto Lavelli \\
\\


 & \hyperlink{page.628}{\em WBI-DDI: Drug-Drug Interaction Extraction using Majority Voting}\\
         & Philippe Thomas, Mariana Neves, Tim Rockt\"{a}schel and Ulf Leser \\
\\

& \hyperlink{page.356}{\em WBI-NER: The impact of domain-specific features on the performance of identifying and classifying mentions of drugs}\\
         & Tim Rockt\"{a}schel, Torsten Huber, Michael Weidlich and Ulf Leser \\
\\


 & \hyperlink{page.636}{\em UMCC\_DLSI: Semantic and Lexical features for detection and classification Drugs in biomedical texts}\\
         & Armando Collazo, Alberto Ceballo, Dennys D. Puig, Yoan Guti\'{e}rrez, Jos\'{e} I. Abreu, Roger P\'{e}rez, Antonio Fern\'{a}ndez Orqu\'{i}n, Andr\'{e}s Montoyo, Rafael Mu\~{n}oz and Franc Camara \\
\\

 & \hyperlink{page.644}{\em NIL\_UCM: Extracting Drug-Drug interactions from text through combination of sequence and tree kernels}\\
         & Behrouz Bokharaeian and ALBERTO DIAZ \\
\\

 & \hyperlink{page.651}{\em UTurku: Drug Named Entity Recognition and Drug-Drug Interaction Extraction Using SVM Classification and Domain Knowledge}\\
         & Jari Bj\"{o}rne, Suwisa Kaewphan and Tapio Salakoski \\
\\

 & \hyperlink{page.660}{\em LASIGE: using Conditional Random Fields and ChEBI ontology}\\
         & Tiago Grego, Francisco Pinto and Francisco M Couto \\
\\

 & \hyperlink{page.667}{\em UWM-TRIADS: Classifying Drug-Drug Interactions with Two-Stage SVM and Post-Processing}\\
         & Majid Rastegar-Mojarad, Richard D. Boyce and Rashmi Prasad \\
\\
\end{tabular}
\newpage
\begin{tabular}{p{20mm}p{138mm}}
\\
\multicolumn{2}{l}{\bf Day 3: Saturday June 15, 2013 (continued)} \\\\
 & \hyperlink{page.675}{\em SCAI: Extracting drug-drug interactions using a rich feature vector}\\
         & Tamara Bobic, Juliane Fluck and Martin Hofmann-Apitius \\
\\

 & \hyperlink{page.684}{\em UColorado\_SOM: Extraction of Drug-Drug Interactions from Biomedical Text using Knowledge-rich and Knowledge-poor Features}\\
         & Negacy Hailu, Lawrence E. Hunter and K. Bretonnel Cohen \\
\\


& \hyperlink{page.290}{\em SemEval-2013 Task 13: Word Sense Induction for Graded and Non-Graded Senses}\\
         & David Jurgens and Ioannis Klapaftis \\
\\


 & \hyperlink{page.689}{\em UoS: A Graph-Based System for Graded Word Sense Induction}\\
         & David Hope and Bill Keller \\
\\


& \hyperlink{page.300}{\em AI-KU: Using Substitute Vectors and Co-Occurrence Modeling For Word Sense Induction and Disambiguation}\\
         & Osman Baskaya, Enis Sert, Volkan Cirik and Deniz Yuret \\
\\


& \hyperlink{page.307}{\em unimelb: Topic Modelling-based Word Sense Induction}\\
         & Jey Han Lau, Paul Cook and Timothy Baldwin \\
\\



\end{tabular}
