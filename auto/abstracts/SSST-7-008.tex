While automatic metrics of translation quality are invaluable for machine translation research, deeper understanding of translation errors require more
 focused evaluations designed to target specific aspects of translation quality.
 We show that Word Sense Disambiguation (WSD)  can be used to evaluate the
 quality of machine translation lexical choice, by applying a standard
 phrase-based SMT system on the SemEval2010 Cross-Lingual WSD task. This case
 study reveals that the SMT system does not perform as well as a WSD system
 trained on the exact same parallel data, and that local context models based on
 source phrases and target n-grams are much weaker representations of context
 than the simple templates used by the WSD system.

