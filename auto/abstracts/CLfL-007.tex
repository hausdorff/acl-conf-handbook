This paper discusses the concept of semantic repetition in literary texts, that is, the recurrence of elements of meaning, possibly in the absence of repeated
 formal elements. A typology of semantic repetition is presented, as well as a
 framework for analysis based on the use of threaded Directed Acyclic Graphs.
 This
 model is applied to the script for the movie Groundhog Day. It is shown first
 that semantic repetition presents a number of traits not found in the case of
 the repetition of formal elements (letters, words, etc.). Consideration of the
 threaded DAG also brings to light several classes of semantic repetition,
 between individual nodes of a DAG, between subDAGs within a larger DAG, and
 between structures of subDAGs, both within and across texts. The model
 presented here provides a basis for the detailed study of additional literary
 texts at the semantic level and illustrates the tractability of the formalism
 used for analysis of texts of some considerable length and complexity.

