The rise of social media has brought computational linguistics in ever-closer contact with bad language: text that defies our expectations about vocabulary,
 spelling, and syntax. This paper surveys the landscape of bad language, and
 offers a critical review of the NLP community's response, which has largely
 followed two paths: normalization and domain adaptation. Each approach is
 evaluated in the context of theoretical and empirical work on computer-mediated
 communication. In addition, the paper presents a quantitative analysis of the
 lexical diversity of social media text, and its relationship to other corpora.

