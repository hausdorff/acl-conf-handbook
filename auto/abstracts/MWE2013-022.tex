A challenging topic of Portuguese language processing is the multifunctional and ambiguous use of the clitic pronoun 'se', which impacts NLP tasks such as
 syntactic parsing, semantic role labeling and machine translation. Aiming to
 give a step  forward towards the automatic disambiguation of 'se', our study
 focuses on the identification of pronominal verbs, which correspond to one of
 the six uses of 'se' as a clitic pronoun, when 'se' is considered a
 constitutive particle of the verb lemma to which is bound, as a multiword unit.
 Our strategy to  identify such verbs is to analyze the results of a corpus
 search to rule out all the other possible uses of 'se'. This process evidenced
 the features needed in a computational lexicon to automatically perform the
 disambiguation task. The availability of the resulting lexicon of pronominal
 verbs on the web enables their inclusion in broader lexical resources, such as
 the Portuguese versions of Wordnet, Propbank and VerbNet. Moreover, it will
 allow the revision of parsers and dictionaries already in use.

