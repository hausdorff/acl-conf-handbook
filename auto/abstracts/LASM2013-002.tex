Does phonological variation get transcribed into social media text?   This paper investigates examples of the phonological variable of
   consonant cluster reduction in Twitter. Not only does this variable
   appear frequently, but it displays the same sensitivity to
   linguistic context as in spoken language. This suggests that when
   social media writing transcribes phonological properties of speech,
   it is not merely a case of inventing orthographic transcriptions.
   Rather, social media displays influence from structural properties
   of the phonological system.

