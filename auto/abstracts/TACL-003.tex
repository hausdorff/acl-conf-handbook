Unsupervised parsing is a difficult task that infants readily perform. Progress has been made on this task using text-based models, but few computational
 approaches have considered how infants might benefit from acoustic cues. This
 paper explores the hypothesis that word duration can help with learning syntax.
 We describe how duration information can be incorporated into an unsupervised
 Bayesian depen- dency parser whose only other source of information is the
 words themselves (without punctuation or parts of speech). Our results,
 evaluated on both adult-directed and child-directed utterances, show that using
 word duration can improve parse quality relative to words-only baselines. These
 results support the idea that acoustic cues provide useful evidence about
 syntactic structure for language-learning infants, and motivate the use of word
 duration cues in NLP tasks with speech.

