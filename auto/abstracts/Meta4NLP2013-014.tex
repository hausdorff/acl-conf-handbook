This work presents the tentative version of the protocol designed for annotation of a Russian metaphor corpus using the rapid annotation tool BRAT.
 The first part of the article is devoted to the procedure of ``shallow''
 annotation in which metaphor-related words are identified according to a
 slightly modified version of the MIPVU procedure. The paper presents the
 results of two reliability tests and the measures of inter-annotator agreement
 obtained in them. Further on, the article gives a brief account of the
 linguistic problems that were encountered in adapting MIPVU to Russian. The
 rest of the first part describes the classes of metaphor-related words and the
 rules of their annotation with BRAT. The examples of annotation show how the
 visualization functionalities of BRAT allow the researcher to describe the
 multifaceted nature of metaphor related words and the complexity of their
 relations.
 The second part of the paper speaks about the annotation of conceptual
 metaphors (the ``deep'' annotation), where formulations of conceptual metaphors
 are inferred from the basic and contextual meanings of metaphor-related words
 from the ``shallow'' annotation, which is expected to make the metaphor
 formulation process more controllable.

