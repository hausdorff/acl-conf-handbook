Recent work has shown that word aligned data can be used to learn a model for reordering source sentences to match the target order. This model learns the
 cost of putting a word immediately before another word and finds the best
 reordering by solving an instance of the Traveling Salesman Problem (TSP).
 However, for efficiently solving the TSP, the model is restricted to pairwise
 features which examine only a pair of words and their neighborhood. In this
 work, we go beyond these pairwise features and learn a model to rerank the
 $n$-best reorderings produced by the TSP model using higher order and
 structural features which help in capturing longer range dependencies. In
 addition to using a more informative set of source side features, we also
 capture target side features indirectly by using the translation score assigned
 to a reordering. Our experiments, involving Urdu-English, show that the
 proposed approach outperforms a state-of-the-art PBSMT system which uses the
 TSP model for reordering by 1.3 BLEU points, and a publicly available
 state-of-the-art MT system, Hiero, by 3 BLEU points.

