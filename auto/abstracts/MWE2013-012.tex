While working on valency lexicons for Czech and English, it was necessary to define treatment of multiword entities (MWEs) with the verb as the central
 lexical unit. Morphological, syntactic and semantic properties of such MWEs had
 to be formally specified in order to create lexicon entries and use them in
 treebank annotation. Such a formal specification has also been used for
 automated quality control of the annotation vs. the lexicon entries. We present
 a corpus-based study, concentrating on multilayer specification of verbal MWEs,
 their properties in Czech and English, and a comparison between the two
 languages using the parallel Czech-English Dependency Treebank (PCEDT). This
 comparison revealed interesting differences in the use of verbal MWEs in
 translation (discovering that such MWEs are actually rarely translated as MWEs,
 at least between Czech and English) as well as some inconsistencies in their
 annotation. Adding MWE-based checks should thus result in better quality
 control of future treebank/lexicon annotation. Since Czech and English are
 typologically different languages, we believe that our findings will also
 contribute to a better understanding of verbal MWEs and possibly their more
 unified treatment across languages.

