We developed an approach to predict the proficiency level of Estonian language learners based on the CEFR guidelines. We performed learner classification by
 studying morpho-syntactic variation and lexical richness in texts produced by
 learners of Estonian as a second language. We show that our features which
 exploit the rich morphology of Estonian by focusing on the nominal case and
 verbal mood are useful predictors for this task. We also show that
 re-formulating the classification problem as a multi-stage cascaded
 classification improves the classification accuracy. Finally, we also studied
 the effect of training data size on classification accuracy and found that more
 training data is beneficial in only some of the cases.

