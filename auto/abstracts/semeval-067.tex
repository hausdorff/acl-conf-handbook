This paper describes the specifications and re-sults of SSA-UO, unsupervised system, pre-sented in SemEval 2013 for Sentiment Analysis in Twitter (Task 2)
 (Wilson et al., 2013). The proposal system includes three phases: data
 preprocessing, contextual word polarity detection and message classification.
 The
 preprocessing phase comprises treatment of emoticon, slang terms, lemmatization
 and POS-tagging. Word polarity detection is carried out taking into account the
 sentiment associated with the context in which it appears. For this, we use a
 new contextual sentiment classification method based on coarse-grained
 word sense disambiguation, using WordNet (Miller, 1995) and a coarse-grained
 sense inventory (sentiment inventory) built up from SentiWordNet (Baccianella
 et al., 2010). Finally, the overall sentiment is determined using a rule-based
 classifier. As it may be observed, the results obtained for Twitter and
 SMS sentiment classification are good considering that our proposal is
 unsupervised.

