Standard phrase-based translation models do not explicitly model context dependence between translation units. As a result, they rely on large phrase
 pairs and target language models to recover contextual effects in translation.
 In this work, we explore n-gram models over Minimal Translation Units (MTUs) to
 explicitly capture contextual dependencies across phrase boundaries in the
 channel model. As there is no single best direction in which contextual
 information should flow, we explore multiple decomposition structures as well
 as  dynamic bidirectional decomposition. The resulting models are evaluated in
 an intrinsic task of lexical selection for MT as well as a full MT system,
 through n-best reranking. These experiments demonstrate that additional
 contextual modeling does indeed benefit a phrase-based system and that the
 direction of conditioning is important. Integrating multiple conditioning
 orders provides consistent benefit, and the most important directions differ by
 language pair.

