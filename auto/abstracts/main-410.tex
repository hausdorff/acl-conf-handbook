We present a new variant of the Syntax-Augmented Machine Translation (SAMT) formalism with a category-coarsening algorithm originally developed for
 tree-to-tree grammars.                          We induce bilingual labels into the
 SAMT
 grammar,
 use
 them for category coarsening, then project back to monolingual labeling as in
 standard SAMT.                          The result is a ``collapsed'' grammar with the
 same
 expressive
 power and format as the original, but many fewer nonterminal labels.  We show
 that the smaller label set provides improved translation scores by 1.14 BLEU on
 two Chinese--English test sets while reducing the occurrence of sparsity and
 ambiguity problems common to large label sets.

