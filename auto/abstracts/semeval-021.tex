This paper describes our system for participating SemEval2013 Task2-B \cite{SemEval2013task2}: Sentiment Analysis in Twitter. Given a message, our
 system classifies whether the message is \emph{positive}, \emph{negative} or
 \emph{neutral} sentiment. It uses a co-occurrence rate model. The training data
 are constrained to the data provided by the task organizers (No other tweet
 data are used). We consider 9 types of features and use a subset of them in our
 submitted system. To see the contribution of each type of features, we do
 experimental study on features by leaving one type of features out each time.
 Results suggest that unigrams are the most important features, bigrams and POS
 tags seem not helpful, and stopwords should be retained to achieve the best
 results. The overall results of our system are promising regarding the
 constrained features and data we use.

