Selecting a set of nonterminals for the synchronous CFGs underlying the hierarchical phrase-based models is usually done on the basis of a monolingual
 resource (like a syntactic parser). However, a standard bilingual resource like
 word alignments is itself rich with reordering patterns that, if clustered
 somehow, might provide labels of different (possibly complementary) nature to
 monolingual labels. In this paper we explore a first version of this idea based
 on a hierarchical decomposition of word alignments into recursive tree
 representations. We identify five clusters of alignment patterns in which the
 children of a node in a decomposition tree are found and employ these five as
 nonterminal labels for the Hiero productions. Although this is our first
 non-optimized instantiation of the idea, our experiments show competitive
 performance with the Hiero baseline, exemplifying certain merits of this novel
 approach.

