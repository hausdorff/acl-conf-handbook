To solve data sparsity problem, recently there has been a trend in discriminative methods of NLP to use representations of lexical items learned
 from unlabeled data as features. In this paper, we investigated the usage of
 word representations learned by neural language models, i.e. word embeddings.
 The direct usage has disadvantages such as large amount of computation,
 inadequacy with dealing word ambiguity and rare-words, and the problem of
 linear non-separability. To overcome these problems, we instead built compound
 features from continuous word embeddings based on clustering. Experiments
 showed that the compound features not only improved the performances on several
 NLP tasks, but also ran faster, suggesting the potential of embeddings.

