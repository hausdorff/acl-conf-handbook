This talk will share stories from recent investigations at the MIT Media Lab in creating technology to recognize and better communicate emotion.  Examples
 include automating facial affect recognition online for sharing media
 experiences, gathering the world's largest sets of natural expressions
 (instead of lab-elicited data) and training machine learning models to predict
 liking of the experience based on expression dynamics throughout the
 experience.   We also have found that most people have difficulty
 discriminating ``peak smiles of frustration'' from ``peak smiles of
 delight'' in static images.  With machine learning and dynamic features, we
 were able to teach the computer to be highly accurate at discriminating these. 
 These kinds of tools can potentially help many people with nonverbal learning
 disabilities, limited vision, social phobia, or autism who find it challenging
 to read the faces of those around them.  I will also share recent findings from
 people wearing physiological sensors 24/7, and how we've been learning about
 connections between the emotion system, sleep and seizures.  Finally, I will
 share some of our newest work related to crowd sourcing cognitive-behavioral
 therapy and computational empathy, where sentiment analysis could be of huge
 benefit.

