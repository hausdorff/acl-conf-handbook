This article provides an in-depth research of machine learning methods for sentiment analysis of Czech social media. Whereas in English, Chinese, or
 Spanish this field has a long history and evaluation datasets for various
 domains are widely available, in case of Czech language there has not yet been
 any systematical research conducted. We tackle this issue and establish a
 common ground for further research by providing a large human-annotated Czech
 social media corpus. Furthermore, we evaluate state-of-the-art supervised
 machine learning methods for sentiment analysis. We explore different
 pre-processing techniques and employ various features and classifiers.
 Moreover, in addition to our newly created social media dataset, we also report
 results on other widely popular domains, such as movie and product reviews. We
 believe that this article will not only extend the current sentiment analysis
 research to another family of languages, but will also encourage competition
 which potentially leads to the production of high-end commercial solutions.

