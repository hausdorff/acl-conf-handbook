Native language identification (NLI) is the task to determine the native language of the author based on an essay written in a second language.             
 NLI is
 often treated as a classification problem.  Most of previous works,
 International Corpus of Learner English (ICLE) is used to develop and evaluate
 algorithms.  In this paper, we use a different data set, TOEFL11, which
 consists of more data (more essays and more languages) and less biased on
 prompts, i.e., topics, of essays.  We demonstrate that even using word level
 n-gram as features, and support vector machine (SVM) as a classifier can yield
 nearly 80\% accuracy. We observe that the accuracy of binary based word level
 n-gram representation (~80\%) is much better than the performance of frequency
 based word level n-gram representation (~17\%).              In addition, for each
 essay
 in TOEFL11, the author's English proficiency level annotated by assessment
 experts is provided.  We also include English proficiency level as a feature
 to train our classifier, but adding this feature slightly decreases accuracy.

