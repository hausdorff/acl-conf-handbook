This paper deals with knowledge-based text processing which aims at an intuitive notion of textual similarity. Entities and relations relevant for a
 particular domain are identified and disambiguated by means of semi-supervised
 machine learning techniques and resulting annotations are applied for computing
 typed-similarity of individual texts.
 
 The work described in this paper particularly shows effects of the mentioned
 processes in the context of the *SEM 2013 pilot task on typed-similarity, a
 part of the Semantic Textual Similarity shared task. The goal is to evaluate
 the degree of semantic similarity between semi-structured records. As the
 evaluation dataset has been taken from Europeana - a collection of records on
 European cultural heritage objects - we focus on computing a semantic distance
 on field author which has the highest potential to benefit from the domain
 knowledge.
 
 Specific features that are employed in our system BUT-TYPED are briefly
 introduced together with a discussion on their efficient acquisition. Support
 Vector Regression is then used to combine the features and to provide a final
 similarity score. The system ranked third on the attribute author among
 15~submitted runs in the typed-similarity task.

