Most NLP tools are applied to text that is different from the kind of text they were evaluated on. Common evaluation practice prescribes significance testing
 across data points in available test data, but typically we only have a single
 test sample. This short paper argues that in order to assess the robustness of
 NLP tools we need to evaluate them on diverse samples, and we consider the
 problem of finding the most appropriate way to estimate the true effect size
 across datasets of our systems over their baselines. We apply meta-analysis and
 show experimentally - by comparing estimated error reduction over observed
 error reduction on held-out datasets - that this method is significantly more
 predictive of success than the usual practice of using macro- or
 micro-averages. Finally, we present a new parametric meta-analysis based on
 non-standard assumptions that seems superior to standard parametric
 meta-analysis.

