This paper evaluates four metaphor identification systems on the 200,000 word VU Amsterdam Metaphor Corpus, comparing results by genre and by sub-class of
 metaphor. The paper then compares the rate of agreement between the systems for
 each genre and sub-class. Each of the identification systems is based,
 explicitly or implicitly, on a theory of metaphor which hypothesizes that
 certain properties are essential to metaphor-in-language. The goal of this
 paper is to see what the success or failure of these systems can tell us about
 the essential properties of metaphor-in-language. The success of the
 identification systems varies significantly across genres and sub-classes of
 metaphor. At the same time, the different systems achieve similar success rates
 on each even though they show low agreement among themselves. This is taken to
 be evidence that there are several sub-types of metaphor-in-language and that
 the ideal metaphor identification system will first define these sub-types and
 then model the linguistic properties which can distinguish these sub-types from
 one another and from non-metaphors.

