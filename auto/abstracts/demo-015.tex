We present KELVIN, an automated system for processing a large text corpus and distilling a knowledge base about persons, organizations, and locations. We
 have tested the KELVIN system on several corpora, including: (a) the TAC KBP
 2012 Cold Start corpus which consists of public Web pages from the University
 of Pennsylvania, and (b) a subset of 26k news articles taken from English
 Gigaword 5th edition.
 
 The system essentially creates a Wikipedia automatically, but one without the
 narrative text. Any text corpus can be used, so the method is not restricted to
 well-known entities.
 
 Our NAACL HLT 2013 demonstration permits a user to interact with a set of
 searchable HTML pages, which are automatically generated from the knowledge
 base. Each page contains information analogous to the semi-structured details
 about an entity that are present in Wikipedia Infoboxes, along with hyperlink
 citations to supporting text.

