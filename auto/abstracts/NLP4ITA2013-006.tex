One of the populations that often needs some form of help to read everyday documents is
 non-native speakers. This paper discusses aid
 at the word and word string levels and focuses
 on the possibility of using translation and
 simplification. Seen from the perspective of
 the non-native as an ever-learning reader, we
 show how translation may be of more harm
 than help in understanding and retaining the
 meaning of a word while simplification holds
 promise. We conclude that if reading everyday
 documents can be considered as a learning
 activity as well as a practical necessity,
 then our study reinforces the arguments that
 defend the use of simplification to make documents
 that non-natives need to read more accessible

