This paper presents a method for sentiment analysis specifically designed to work with Twitter data (tweets), taking into account 
  their structure, length and specific language. The approach employed makes it
 easily extendible to other languages and makes it able to process tweets in
 near real time. 
 
  The main contributions of this work are: a) the pre-processing of tweets to
 normalize the language and generalize the vocabulary employed to express
 sentiment; b) the use minimal linguistic processing, which makes the approach
 easily portable to other languages; c) the inclusion of higher order n-grams to
 spot modifications in the polarity of the sentiment expressed; d) the use of
 simple heuristics to select features to be employed; e) the application of
 supervised learning using a simple Support Vector Machines linear classifier on
 a set of realistic data. We show that using the training models generated with
 the method described we can improve the sentiment classification performance,
 irrespective of the domain and distribution of the test sets.

