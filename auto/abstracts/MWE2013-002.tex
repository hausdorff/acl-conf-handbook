We present an experimental study of how different features help measuring the idiomaticity of noun+verb (NV) expressions in Basque. After testing several
 techniques for quantifying the four basic properties of multiword expressions
 or MWEs (institutionalization, semantic non-compositionality, morphosyntactic
 fixedness and lexical fixedness), we test different combinations of them for
 classification into idioms and collocations, sing Machine Learning (ML) and
 feature selection. Key aspects for inflectional and agglutinative languages are
 included. The results show the major role of distributional similarity, which
 measures compositionality, in the extraction and classification of MWEs,
 especially, as expected, in the case of idioms. Even though cooccurrence and
 some aspects of morphosyntactic flexibility contribute to this task in a more
 limited measure, ML experiments make benefit of these sources of knowledge,
 allowing to improve the results obtained using exclusively distributional
 similarity features.

