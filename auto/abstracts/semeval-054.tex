The fast development of Social Media made it possible for people to no loger remain mere spectators to the events that happen in the world, 
  but become part of them, commenting on their developments and the entities
 involved, 
  sharing their opinions and distributing related content. 
  This phenomenon is of high importance to news monitoring systems, whose aim is
 to obtain an informative snapshot of media events and related comments.
 
  This paper presents the strategies employed in the OPTWIMA participation to
 SemEval 2013 Task 2-Sentiment
 Analysis in Twitter. The main goal was to evaluate the best settings for a
 sentiment analysis component
 to be added to the online news monitoring system. 
 
 We describe the approaches used in the competition and the
 additional experiments performed combining different datasets for training,
 using or not slang replacement and generalizing sentiment-bearing terms by
 replacing them with unique labels. The results regarding tweet classification
 are promising and show that sentiment generalization can be an effective
 approach for tweets and that SMS language is difficult to tackle, even when
 specific normalization resources are employed.

