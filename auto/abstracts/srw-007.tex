We present our work in generating Karmina, an old Malay poetic form for Indonesian language. Karmina is a poem with two lines that consists of a hook
 (sampiran) on the first line and a message on the second line. One of the
 unique aspects of Karmina is in the absence of discourse relation between its
 hook and message. We approached the problem by generating the hooks and the
 messages in separate processes using predefined schemas and a manually built
 knowledge base. The Karminas were produced by randomly pairing the messages
 with the hooks, subject to the constraints imposed on the rhymes and on the
 structure similarity. Syllabifications were performed on the cue words of the
 hooks and messages to ensure the generated pairs have matching rhymes. We were
 able to generate a number of positive examples while still leaving room for
 improvement, particularly in the generation of the messages, which currently
 are still limited, and in filtering the negative results.

