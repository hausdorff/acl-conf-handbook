This paper presents preliminary results of using authorship identification methods for the detection of sockpuppeteering in Wikipedia. Sockpuppets are
 fake accounts created by malicious users to bypass Wikipedia's regulations. Our
 dataset is composed of the comments made by the editors on the talk pages. To
 overcome the limitations of the short lengths of these comments, we use an
 voting scheme to combine predictions made on individual user entries. We show
 this approach is promising and that it can be a viable alternative to the
 current human process that Wikipedia uses to resolve suspected sockpuppet
 cases.

