Bibliometric measures are commonly used to estimate the popularity and the impact of published research. Existing bibliometric measures
 provide ``quantitative'' indicators of how good a published paper is.
 This does not necessarily reflect the ``quality'' of the work
 presented in the paper. For example, when h-index is computed
 for a researcher, all incoming citations are treated equally, ignoring
 the fact that some of these citations might be negative. In this
 paper, we propose using NLP to add a ``qualitative'' aspect to
 biblometrics. We analyze the text that accompanies citations in
 scientific articles (which we term citation context). We
 propose supervised methods for identifying citation text and analyzing
 it to determine the purpose (i.e. author intention) and the polarity
 (i.e. author sentiment) of citation.

