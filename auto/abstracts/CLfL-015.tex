Scholars of Chinese literature note that China's tumultuous literary history in the 20th century centered around the uncomfortable tensions between tradition
 and modernity. In this corpus study, we develop and automatically extract three
 features to show that the classical character of Chinese poetry decreased
 across
 the century. We also find that Taiwan poets constitute a surprising exception
 to the trend, demonstrating an unusually strong connection to classical diction
 in their work as late as the '50s and '60s.

