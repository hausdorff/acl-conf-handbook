In this paper, we study the problem of automatic enrichment of a morphologically underspecified treebank for Arabic, a morphologically rich
 language. We show that we can map from a tagset of size six to one with 485
 tags at an accuracy rate of 94\%-95\%. We can also identify the unspecified
 lemmas in the treebank with an accuracy over 97\%. Furthermore, we demonstrate
 that using our automatic annotations improves the performance of a
 state-of-the-art Arabic morphological tagger.  Our approach combines a variety
 of techniques from corpus-based statistical models to linguistic rules that
 target specific phenomena.  These results suggest that the cost of treebanking
 can be reduced by designing underspecified treebanks that can be subsequently
 enriched automatically.

