Stylometric analysis of prose is typically limited to classification tasks such as authorship
 attribution. Since the models used are typically
 black boxes, they give little insight into
 the stylistic differences they detect. In this
 paper, we characterize two prose genres syntactically:
 chick lit (humorous novels on the
 challenges of being a modern-day urban female)
 and high literature. First, we develop
 a top-down computational method based on
 existing literary-linguistic theory. Using an
 off-the-shelf parser we obtain syntactic structures
 for a Dutch corpus of novels and measure
 the distribution of sentence types in chick-lit
 and literary novels. The results show that literature
 contains more complex (subordinating)
 sentences than chick lit. Secondly, a bottom-up
 analysis is made of specific morphological and
 syntactic features in both genres, based on the
 parser's output. This shows that the two genres
 can be distinguished along certain features.
 Our results indicate that detailed insight into
 stylistic differences can be obtained by combining
 computational linguistic analysis with
 literary theory.

