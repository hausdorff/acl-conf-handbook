We propose a method to learn succinct hierarchical linguistic descriptions of visual datasets, which allow for improved navigation efficiency in image
 collections. Classic exploratory data analysis methods, such as agglomerative
 hierarchical clustering, only provide a means of obtaining a tree-structured
 partitioning of the data. This requires the user to go through the images
 first, in order to reveal the semantic relationship between the different
 nodes. On the other hand, in this work we propose to learn a hierarchy of
 linguistic descriptions, referred to as attributes, which allows for a textual
 description of the semantic content that is captured by the hierarchy. Our
 approach is based on a generative model, which relates the attribute
 descriptions associated with each node, and the node assignments of the data
 instances, in a probabilistic fashion. We furthermore use a nonparametric
 Bayesian prior, known as the tree-structured stick breaking process, which
 allows for the structure of the tree to be learned in an unsupervised fashion.
 We also propose appropriate performance measures, and demonstrate superior
 performance compared to other hierarchical clustering algorithms.

