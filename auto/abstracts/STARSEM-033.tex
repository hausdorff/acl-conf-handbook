Social scientists are increasingly using the vast amount of text available on social media to measure variation in happiness and other psychological states.
 Such studies count words deemed to be indicators of happiness and track how the
 word frequencies change across locations or time. This ``word count'' approach
 is simple and scalable, yet often picks up false signals, as words can appear
 in different contexts and take on different meanings. We characterize the types
 of errors that occur using the word count approach, and find lexical ambiguity
 to be the most prevalent. We then show that one can reduce error with a simple
 refinement to such lexica by automatically eliminating highly ambiguous words.
 The resulting refined lexica improve precision as measured by human judgments
 of word occurrences in Facebook posts.

