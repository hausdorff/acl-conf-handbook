To avoid a sarcastic message being understood in its unintended literal meaning, in microtexts such as messages on Twitter.com sarcasm is often
 explicitly marked with the hashtag `\\#sarcasm'. We collected a training corpus
 of about 78 thousand Dutch tweets with this hashtag. Assuming that the human
 labeling is correct (annotation of a sample indicates that about 85\\% of these
 tweets are indeed sarcastic), we train a machine learning classifier on the
 harvested examples, and apply it to a test set of a day's stream of 3.3 million
 Dutch tweets. Of the 135 explicitly marked tweets on this day, we detect 101
 (75\\%) when we remove the hashtag. We annotate the top of the ranked list of
 tweets most likely to be sarcastic that do not have the explicit hashtag. 30\\%
 of the top-250 ranked tweets are indeed sarcastic. Analysis shows that sarcasm
 is often signalled by hyperbole, using intensifiers and exclamations; in
 contrast, non-hyperbolic sarcastic messages often receive an explicit marker.
 We hypothesize that explicit markers such as hashtags are the digital
 extralinguistic equivalent of non-verbal expressions that people employ in live
 interaction when conveying sarcasm.

