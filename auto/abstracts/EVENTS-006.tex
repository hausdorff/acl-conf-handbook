Understanding the event structure of sentences and whole documents is an important step in being able to extract meaningful information from the text.
 Our task is the identification of phenotypes, specifically, pneumonia, from
 clinical narratives. In this paper, we consider the importance of identifying
 the change of state for events, in particular, events that measure and compare
 multiple states across time. Change of state is important to the clinical
 diagnosis of pneumonia; in the example ``there are bibasilar opacities that
 are unchanged'', the presence of bibasilar opacities alone may suggest
 pneumonia, but not when they are unchanged, which suggests the need to modify
 events with change of state information. Our corpus is comprised of chest X-
 ray reports, where we find many descriptions of change of state comparing the
 volume and density of the lungs and surrounding areas. We propose an annotation
 schema to capture this information as a tuple of <location, attribute, value,
 change-of-state, time-reference>.

