This position paper argues the need for a comprehensive corpus of online book re-sponses. Responses to books (in traditional reviews, book blogs, on
 booksellers' sites, etc.) are important for understanding how readers
 understand literature and how literary works become popular. A sufficiently
 large, varied and representative corpus of online responses to books will
 facilitate research into these processes. This corpus should include context
 information about the responses and should remain open to additional material.
 Based on a pilot study for the creation of a corpus of Dutch online book
 response, the paper shows how linguistic tools can find differences in word
 usage between responses from various sites. They can also reveal response type
 by clustering responses based on usage of either words or their POS-tags, and
 can show the sentiments expressed in the responses. LSA-based similarity
 between book fragments and response may be able to reveal the book fragments
 that most affected readers. The pa-per argues that a corpus of book responses
 can be an important instrument for research into reading behavior, reader
 response, book reviewing and literary appreciation.

