This paper describes a temporal expression identification and normalization system, ManTIME, developed for the TempEval-3 challenge. The identification
 phase combines the use of conditional random fields along with a
 post-processing identification pipeline, whereas the normalization phase is
 carried out using NorMA, an open-source rule-based temporal normalizer. We
 investigate the performance variation with respect to different feature types.
 Specifically, we show that the use of WordNet-based features in the
 identification task negatively affects the overall performance, and that there
 is no statistically significant difference in using gazetteers, shallow parsing
 and propositional noun phrases labels on top of the morphological features. On
 the test data, the best run achieved 0.95 (P), 0.85 (R) and 0.90 (F1) in the
 identification phase. Normalization accuracies are 0.84 (type attribute) and
 0.77 (value attribute). Surprisingly, the use of the silver data (alone or in
 addition to the gold annotated ones) does not improve the performance.

