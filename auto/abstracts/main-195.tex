Language variations are generally known to have a severe impact on the performance of Human Language Technology Systems. In order to predict or
 improve system performance, a thorough investigation into these variations,
 similarities and dissimilarities, is required. Distance measures have been used
 in several applications of speech processing to analyze different varying
 speech attributes. However, not much work has been done on language distance
 measures, and even less work has been done involving South African languages.
 This study explores two methods for measuring the linguistic distance of six
 South African languages. It concerns a text based method, (the Levenshtein
 Distance), and an acoustic approach using extracted mean pitch values. The
 Levenshtein distance uses parallel word transcriptions from all six languages
 with as little as 144 words, whereas the pitch method is text-independent and
 compares mean language pitch differences. Cluster analysis resulting from the
 distance matrices from both methods correlates closely with human perceptual
 distances and existing literature about the six languages.

