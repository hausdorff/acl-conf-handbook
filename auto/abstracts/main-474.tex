The frequency of words and syntactic constructions has been observed to have a substantial effect on language processing. This begs the question of what
 causes certain constructions to be more or less frequent. A theory of grounding
 (Phillips, 2010) would suggest that cognitive limitations might cause languages
 to develop frequent constructions in such a way as to avoid processing costs.
 This paper studies how current theories of working memory fit into theories of
 language processing and what influence memory limitations may have over reading
 times. Measures of such limitations are evaluated on eye-tracking data and the
 results are compared with predictions made by different theories of processing.

