In the field of Intelligent User Interfaces, Spoken Dialogue Systems (SDSs) play a key role as speech represents a true intuitive means of human
 communication. Deriving information about its quality can help rendering SDSs
 more user-adaptive. Work on automatic estimation of subjective quality usually
 relies on statistical models. To create those, manual data annotation is
 required, which may be performed by actual users or by experts. Here, both
 variants have their advantages and drawbacks. In this paper, we analyze the
 relationship between user and expert ratings by investigating models which
 combine the advantages of both types of ratings. We explore two novel
 approaches using statistical classification methods and evaluate those with a
 preexisting corpus providing user and expert ratings. After analyzing the
 results, we eventually recommend to use expert ratings instead of user ratings
 in general.

