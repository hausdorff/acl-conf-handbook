Word Sense Disambiguation (WSD) approaches have reported good accuracies in recent years. However, these approaches can be classified as weak AI systems.
 According to the classical definition, a strong AI based WSD system should
 perform the task of sense disambiguation in the same manner and with similar
 accuracy as human beings. In order to accomplish this, a detailed understanding
 of the human techniques employed for sense disambiguation is necessary. Instead
 of building yet another WSD system that uses contextual evidence for sense
 disambiguation, as has been done before, we have taken a step back - we have
 endeavored to discover the cognitive faculties that lie at the very core of the
 human sense disambiguation technique.
 
 In this paper, we present a hypothesis regarding the cognitive sub-processes
 involved in the task of WSD. We support our hypothesis using the experiments
 conducted through the means of an eye-tracking device. We also strive to find
 the levels of difficulties in annotating various classes of words, with senses.
 We believe, once such an in-depth analysis is performed, numerous insights can
 be gained to develop a robust WSD system that conforms to the principle of
 strong AI.

