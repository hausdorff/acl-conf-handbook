This study focuses on modeling discourse coherence in the context of automated assessment of spontaneous speech from non-native speakers. Discourse coherence
 has always been used as a key metric in human scoring rubrics for various
 assessments of spoken language. However, very little research has been done to
 assess a speaker's coherence in automated speech scoring systems. To address
 this, we present a corpus of spoken responses that has been annotated for
 discourse coherence quality. Then, we investigate the use of several features
 originally developed for essays to model coherence in spoken responses. An
 analysis on the annotated corpus shows that the prediction accuracy for human
 holistic scores of an automated speech scoring system can be improved by around
 10% relative after the addition of the coherence features.  Further experiments
 indicate that a weighted F-Measure of 73% can be achieved for the automated
 prediction of the coherence scores.

