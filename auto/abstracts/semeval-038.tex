Named entity recognition (NER) systems are often based on machine learning techniques to reduce the labor-intensive development of hand-crafted
 extraction rules and domain-dependent dictionaries. Nevertheless,
 time-consuming feature engineering is often needed to achieve state-of-the-art
 performance. In this study, we investigate the impact of such domain-specific
 features on the performance of recognizing and classifying mentions of
 pharmacological substances. We compare the performance of a system based on
 general features, which have been successfully applied to a wide range of NER
 tasks, with a system that additionally uses features generated from the output
 of an existing chemical NER tool and a collection of domain-specific resources.
 We demonstrate that acceptable results can be achieved with the former system.
 Still, our experiments show that using domain-specific features outperforms
 this general approach. Our system ranked first in the SemEval-2013 Task 9.1:
 Recognition and classification of pharmacological substances.

