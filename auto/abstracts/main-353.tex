One way to improve the accuracy of automatic speech recognition (ASR) is to use discriminative language modeling (DLM), which enhances discrimination by
 learning where the ASR hypotheses deviate from the uttered sentences. However,
 DLM requires large amounts of ASR output to train. Instead, we can simulate the
 output of an ASR system, in which case the training becomes semi-supervised.
 The advantage of using simulated hypotheses is that we can generate as many
 hypotheses as we want provided that we have enough text material. In typical
 scenarios, transcribed in-domain data is limited but large amounts of
 out-of-domain (OOD) data is available. In this study, we investigate how
 semi-supervised training performs with OOD data. We find out that OOD data can
 yield improvements comparable to in-domain data.

