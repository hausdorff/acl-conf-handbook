Addressee detection (AD) is an important problem for dialog systems in human-human-computer scenarios (contexts involving multiple people and a
 system) because system-directed speech must be distinguished from
 human-directed speech. Recent work on AD (Shriberg et al., 2012) showed good
 results using prosodic and lexical features trained on in-domain data.
 In-domain data, however, is expensive to collect for each new domain. In this
 study we focus on lexical models and investigate how well out-of-domain data
 (either outside the domain, or from single-user scenarios) can fill in for
 matched in-domain data. We find that human-addressed speech can be modeled
 using out-of-domain conversational speech transcripts, and that human-computer
 utterances can be modeled using single-user data: the resulting AD system
 outperforms a system trained only on matched in-domain data. Further gains (up
 to a 4\% reduction in equal error rate) are obtained when in-domain and
 out-of-domain models are interpolated. Finally, we examine which parts of an
 utterance are most useful. We find that the first 1.5 seconds of an utterance
 contain most of the lexical information for AD, and analyze which lexical items
 convey this. Overall, we conclude that the H-H-C scenario can be approximated
 by combining data from H-C and H-H scenarios only.

