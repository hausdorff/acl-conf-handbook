Morphological tokenization has been used in machine translation for morphologically complex languages to reduce lexical sparsity. Unfortunately,
 when translating into a morphologically
 complex language, recombining segmented tokens to generate original word forms
 is not a trivial task, due to morphological, phonological and orthographic
 adjustments that occur during tokenization. We review a number of
 detokenization schemes for Arabic, such as rule-based and table-based
 approaches and show their limitations. We then propose a novel detokenization
 scheme that uses a character-level discriminative string transducer to predict
 the original form of a segmented word. In a comparison to a state-of-the-art
 approach, we demonstrate slightly better detokenization error rates, without
 the need for any hand-crafted rules. We also demonstrate the effectiveness of
 our approach in an English-to-Arabic translation task.

