\documentclass[letterpaper]{article}
\usepackage[left=1in,right=1in,top=1in,bottom=1in]{geometry}
\usepackage{url}
\usepackage{hyperref}
\usepackage{helvet}

\author{Ulrich Germann}
\title{How to Produce Your Proceedings Outside of the START System}

\begin{document}
\maketitle
\section{Introduction}
This document describes briefly how to produce conference or workshop
proceedings for ACL conferences bypassing the START system for
proceedings generation.
\section{Instructions}
I assume that you are in the directory that you have designated as the
root directory for this project.

\section{Required Data and Software}

\begin{itemize}
\item The tarball with everything init from the START system.
\begin{enumerate}
\item Log into the START conference manager console.
\item Go to Conference Program --- ACLPUB.
\item Go to the tab 'Generate'.
\item Under ``Generate Buttons'', click on ``ALL''. 
\item Wait till the page reloads. Generating the tar ball may take a moment.
\item Click on ``Download the final tarball proceedings.tgz''.
\item We're finished with START. Log out.
\item Unpack the tar ball in your working directory: 
{\tt tar xzf proceedings.tgz}
\end{enumerate}

\item The aclpub software. 
\begin{itemize}
\item Get it through cvs:
{\tt cvs -d :pserver:anoncvs@ftp.clsp.jhu.edu:/aclpub checkout aclpub}
\item {\tt export ACLPUB=`pwd`/aclpub.}
\item {\tt cp \${ACLPUB}/make/Makefile\_bookchair proceedings Makefile}
\end{itemize}

\item Follow the instructions in {\tt aclpub/doc/bookchairs/index.html}. If you downloaded the tar ball from START as described above, you should be able to skip all the steps before {\tt make get-order}.

\item Try {\tt make get-oder}. At this point, perl might complain about not finding the perl
  module Text-PDF. You can get it here:
  \url{http://search.cpan.org/~mhosken/Text-PDF-0.29a/}.
\begin{itemize}
\item Click on `Download' to get a tarball. 
\item Unpack the tarball ({\tt tar xzf ...}) 
\item {\tt ln -s ln -s Text-PDF-0.29/lib/Text .}\\
(Obviously the module version number may be different.)
\end{itemize}

\item Edit the file 'order' to your needs. 
\item Follow the rest of the instructions for producing the proceedings.
\end{itemize}

\end{document}
