\documentclass{article}
\usepackage[T1]{fontenc}
\usepackage[utf8]{inputenc}
\usepackage{times}
\usepackage{url}
\title{How to Create Conference Handbooks for ACL Conferences}
\author{Ulrich Germann}

\begin{document}
\maketitle
\section{Introductory remarks}
This document explains how to produce conference handbooks for ACL
conferences from information that can be obtained from the START
conference management system. The process described here is very
unpolished. It would benefit from a proper document class for
conference handbooks (with options for half-letter and A5) instead of
hacked-in changes to the standard book class and improved layout for
chapters, sections etc. Moreover, some of the scripts have
conference-specific info hard-coded (e.g., the day and time of the
poster session). I'm releasing the code in the hope that over time it
will grow into a neat little package that requires less manual work.

\begin{flushright}
Ulrich Germann
\end{flushright}

\section{What you need}
{\bfseries For each conference part} (i.e., main session, workshops, co-located conferences, etc.)
\begin{itemize}
\item the files \verb$proceedings/final/*/*_metadata.txt$ from the ACLPUB tar ball.
\item the file {\tt proceedings/order} from the ACLPUB tar ball.
\item possibly some latex packages that you may not have
  installed. They are included in the directory {\tt texmf}. Copy or
  {\tt rsync} them to your local texmf directory (usually
  \url{~/texmf}) if latex complains about missing packages or unknown
  commands, and don't forget to run {\tt texconfig rehash} afterwards
  to announce to latex that you've installed new packages.
\end{itemize}

The document {\tt aclpub-setup.pdf} explains
how to get the aclpub tar ball out of START.

\section{Scripts provided}
The following scripts are provided with this package.
\begin{description}
\item[{\bfseries meta2bibtex.py}] reads the \verb$*_metadata.txt$
  files and produces a rudimentary .bib file with author names and
  paper titles, and .tex files with the paper abstract for inclusion
  into latex via the \verb$\input$ command.
\item[{\bfseries order2schedule.perl}] reads the {\tt order} files and
  produces various files for inclusion into latex via the
  \verb$\input$ command. This script was tailored to NAACL 2012 and
  will probably need to be adapted to your specific order file. The
  script is picky about the difference between '=' and '+' in the file
  {\tt order}!
\item[{\bfseries starsem.order2schedule.perl}] is a special version of
  order2schedule.perl for parallel sessions with non-synchronized
  paper slots.
\item[{\bfseries singletrack.order2schedule.perl}] is version of
  order2schedule.perl for single-track events such as workshops.

\item[{\bfseries fix-index.perl}] is a script that fixes the .idx file
  produced by running latex on the document. It removes accents for
  proper sorting of the author index. It is called via the makefile.
\end{description}

\section{Handbook assembly}
Most of the skeleton of the handbook is still hand-coded; you'll need
to look at main.tex and included files and adapt them to your needs.

Some parts, however, are automatically generated (basically everyhing
in the directory {\tt auto}).

For each conference part, run 

\begin{verbatim}
meta2bibtex.py <fdir> <tag>
\end{verbatim}

where {\tt fdir} is the directory {\tt final} in the aclpub tar ball
for the respective conference part, and {\tt tag} a tag that you
assign to the conference part (e.g., main, ws1, demos, etc.).

Also run 

\begin{verbatim}
cat /wherever/it/is/order | order2schedule.perl <tag>
\end{verbatim}

to produce various files that can then be \verb$\input$ into the main
latex document.


For all the gory details beyond this, consult the file {\tt handbook.tex}
and files included from there. 
\end{document}


